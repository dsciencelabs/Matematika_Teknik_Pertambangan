% Options for packages loaded elsewhere
\PassOptionsToPackage{unicode}{hyperref}
\PassOptionsToPackage{hyphens}{url}
%
\documentclass[
]{book}
\usepackage{amsmath,amssymb}
\usepackage{iftex}
\ifPDFTeX
  \usepackage[T1]{fontenc}
  \usepackage[utf8]{inputenc}
  \usepackage{textcomp} % provide euro and other symbols
\else % if luatex or xetex
  \usepackage{unicode-math} % this also loads fontspec
  \defaultfontfeatures{Scale=MatchLowercase}
  \defaultfontfeatures[\rmfamily]{Ligatures=TeX,Scale=1}
\fi
\usepackage{lmodern}
\ifPDFTeX\else
  % xetex/luatex font selection
\fi
% Use upquote if available, for straight quotes in verbatim environments
\IfFileExists{upquote.sty}{\usepackage{upquote}}{}
\IfFileExists{microtype.sty}{% use microtype if available
  \usepackage[]{microtype}
  \UseMicrotypeSet[protrusion]{basicmath} % disable protrusion for tt fonts
}{}
\makeatletter
\@ifundefined{KOMAClassName}{% if non-KOMA class
  \IfFileExists{parskip.sty}{%
    \usepackage{parskip}
  }{% else
    \setlength{\parindent}{0pt}
    \setlength{\parskip}{6pt plus 2pt minus 1pt}}
}{% if KOMA class
  \KOMAoptions{parskip=half}}
\makeatother
\usepackage{xcolor}
\usepackage{longtable,booktabs,array}
\usepackage{calc} % for calculating minipage widths
% Correct order of tables after \paragraph or \subparagraph
\usepackage{etoolbox}
\makeatletter
\patchcmd\longtable{\par}{\if@noskipsec\mbox{}\fi\par}{}{}
\makeatother
% Allow footnotes in longtable head/foot
\IfFileExists{footnotehyper.sty}{\usepackage{footnotehyper}}{\usepackage{footnote}}
\makesavenoteenv{longtable}
\usepackage{graphicx}
\makeatletter
\newsavebox\pandoc@box
\newcommand*\pandocbounded[1]{% scales image to fit in text height/width
  \sbox\pandoc@box{#1}%
  \Gscale@div\@tempa{\textheight}{\dimexpr\ht\pandoc@box+\dp\pandoc@box\relax}%
  \Gscale@div\@tempb{\linewidth}{\wd\pandoc@box}%
  \ifdim\@tempb\p@<\@tempa\p@\let\@tempa\@tempb\fi% select the smaller of both
  \ifdim\@tempa\p@<\p@\scalebox{\@tempa}{\usebox\pandoc@box}%
  \else\usebox{\pandoc@box}%
  \fi%
}
% Set default figure placement to htbp
\def\fps@figure{htbp}
\makeatother
\setlength{\emergencystretch}{3em} % prevent overfull lines
\providecommand{\tightlist}{%
  \setlength{\itemsep}{0pt}\setlength{\parskip}{0pt}}
\setcounter{secnumdepth}{5}
\usepackage{booktabs}
\usepackage{lscape}

\usepackage[]{natbib}
\bibliographystyle{apalike}
\usepackage{bookmark}
\IfFileExists{xurl.sty}{\usepackage{xurl}}{} % add URL line breaks if available
\urlstyle{same}
\hypersetup{
  pdftitle={Matematika Teknik Pertambangan},
  pdfauthor={Bakti Siregar, M.Sc},
  hidelinks,
  pdfcreator={LaTeX via pandoc}}

\title{Matematika Teknik Pertambangan}
\author{Bakti Siregar, M.Sc}
\date{2024-07-09}

\begin{document}
\maketitle

{
\setcounter{tocdepth}{1}
\tableofcontents
}
\chapter*{Kata Pengantar}\label{kata-pengantar}
\addcontentsline{toc}{chapter}{Kata Pengantar}

Dengan mengucap syukur kepada Tuhan Yang Maha Esa, saya sangat senang dapat mempersembahkan eBook ``Matematika Tambang'' kepada para pembaca yang budiman. Buku ini merupakan hasil dari proses panjang, penelitian mendalam, serta dedikasi yang tiada henti. Harapan saya, eBook ini dapat memberikan manfaat yang signifikan bagi para pembaca, baik itu akademisi, praktisi, maupun masyarakat umum yang memiliki minat pada bidang ini.

Buku ini disusun dengan tujuan memberikan wawasan dan pengetahuan yang komprehensif mengenai konsep-konsep matematika yang relevan dan aplikatif dalam bidang pertambangan. Melalui buku ini, pembaca diharapkan dapat memahami konsep dasar kalkulus, aljabar linear, statistik, dan probabilitas yang diterapkan dalam pemodelan dan analisis data pertambangan.

\section*{Ringkasan Materi}\label{ringkasan-materi}
\addcontentsline{toc}{section}{Ringkasan Materi}

Adapun isi pembelajaran dalam modul ini adalah sebagai berikut:

\begin{itemize}
\tightlist
\item
  Kalkulus

  \begin{itemize}
  \tightlist
  \item
    Pengantar Matematika Tambang
  \item
    Bilangan Real
  \item
    Konsep Fungsi
  \item
    Konsep Limit Fungsi
  \item
    Konsep Turunan Fungsi
  \item
    Penggunaan Turunan
  \item
    Konsep Integral Fungsi
  \item
    Penggunaan Integral
  \item
    Konsep Fungsi Transenden
  \end{itemize}
\item
  Aljabar Linear

  \begin{itemize}
  \tightlist
  \item
    Matriks
  \item
    Determinan
  \item
    Sistem Persamaan Linear
  \item
    Eigenvalue
  \item
    Eigenvector
  \item
    Aplikasi dalam Pertambangan
  \end{itemize}
\item
  Statistik

  \begin{itemize}
  \tightlist
  \item
    Pengantar Statistik
  \item
    Distribusi Probabilitas
  \item
    Inferensi Statistik
  \item
    Uji Hipotesis
  \end{itemize}
\item
  Pemodelan Matematika

  \begin{itemize}
  \tightlist
  \item
    Model Linear
  \item
    Model Nonlinear
  \item
    Simulasi Monte Carlo
  \item
    Penerapan SMC dalam Pertambangan
  \item
    Metode Iteratif
  \item
    Aplikasi MI dalam Pertambangan
  \end{itemize}
\item
  Analisis Data Tambang

  \begin{itemize}
  \tightlist
  \item
    Teknik Analisis
  \item
    Pengolahan Data
  \end{itemize}
\item
  Optimisasi dalam Pertambangan

  \begin{itemize}
  \tightlist
  \item
    Metode Optimisasi
  \item
    Studi Kasus
  \end{itemize}
\end{itemize}

\section*{Penulis}\label{penulis}
\addcontentsline{toc}{section}{Penulis}

\begin{itemize}
\tightlist
\item
  \textbf{\href{https://www.linkedin.com/in/dsciencelabs/}{Bakti Siregar, M.Sc., CDS}} bekerja sebagai Dosen di Prodi Sains Data Institut Teknologi Sains Bandung. Beliau adalah meraih gelar Magister-nya dari Departemen Matematika Terapan (Applied Mathematics) National Sun Yat Sen University, Taiwan. Selain mengajar beliau juga pernah bekerja sebagai \textbf{Data Scientist Freelance} di perusahaan-perusahaan ternama seperti \href{https://www.jne.co.id/id/beranda}{JNE}, \href{https://www.samoragroup.co.id/home/en}{Samora Group}, \href{https://www.pertamina.com/}{Pertamina}, dan saat ini masih aktif sebagai \textbf{Data Scientist Freelance} di \href{PT.\%20Green\%20City\%20Traffic}{PT. Green City Traffic}. Beliau memiliki antusiasme khusus dalam mengerjakan proyek (mengajar) Big Data Analytics, Machine Learning, Optimisasi, dan Analisis Time Series di bidang keuangan dan investasi. Keahlian utama yang dimilikinya adalah bahasa pemrograman Statistik seperti R Studio dan Python. Beliau juga sudah terbiasa dalam mengaplikasikan sistem basis data MySQL/NoSQL untuk manajemen data, serta mahir dalam menggunakan tools Big Data seperti Spark dan Hadoop. Beberapa project beliau dapat dilihat di link berikut: \href{https://rpubs.com/dsciencelabs}{Rpubs}, \href{https://github.com/dsciencelabs}{Github}, \href{https://dsciencelabs.github.io/web/index.html}{Website}, dan \href{https://www.kaggle.com/baktisiregar/code}{Kaggle}.
\end{itemize}

\section*{Ucapan Terima Kasih}\label{ucapan-terima-kasih}
\addcontentsline{toc}{section}{Ucapan Terima Kasih}

Proses penulisan eBook ini tidak terlepas dari dukungan berbagai pihak. Saya ingin mengucapkan terima kasih kepada:

\begin{itemize}
\tightlist
\item
  \textbf{Keluarga} yang selalu memberikan dukungan moral dan semangat tanpa henti.
\item
  \textbf{Rekan-rekan dan Kolaborator} yang telah memberikan masukan, saran, dan kritik yang konstruktif.
\item
  Institusi dan Organisasi, Khususnya \textbf{\href{https://itsb.ac.id/}{ITSB}} yang telah menyediakan sumber daya dan fasilitas yang diperlukan selama proses penelitian dan penulisan.
\end{itemize}

Saya berharap eBook ini dapat menjadi referensi yang bermanfaat dan memberikan inspirasi serta pengetahuan baru bagi pembaca. Semoga Ebook ini dapat memenuhi ekspektasi dan kebutuhan para pembaca dan semoga ilmu yang disampaikan dapat bermanfaat bagi semua.

\section*{Masukan \& Saran}\label{masukan-saran}
\addcontentsline{toc}{section}{Masukan \& Saran}

Semua masukan dan tanggapan Anda sangat berarti bagi kami untuk memperbaiki Ebook Matematika Tambang ini kedepannya. Bagi para pembaca/pengguna yang ingin menyampaikan masukan dan tanggapan, dipersilahkan melalui kontak dibawah ini!

\textbf{Email:} \href{mailto:dsciencelabs@outlook.com}{\nolinkurl{dsciencelabs@outlook.com}}

\chapter{Pengantar Matematika Teknik Pertambangan \{\#Pengantar\_Matematika\_Teknik\_Pertambangan)}\label{pengantar-matematika-teknik-pertambangan-pengantar_matematika_teknik_pertambangan}

\section{Praktikum}\label{praktikum}

\chapter{Bab2}\label{bab2}

\section{Praktikum}\label{praktikum-1}

\chapter{Konsep Fungsi}\label{Konsep_Fungsi}

\section{Praktikum}\label{praktikum-2}

\chapter{Konsep Limit Fungsi}\label{Konsep_Limit_Fungsi}

\section{Praktikum}\label{praktikum-3}

\chapter{Konsep Turunan}\label{Konsep_Turunan}

\section{Praktikum}\label{praktikum-4}

\chapter{Penggunaan Turunan Fungsi}\label{Penggunaan_Turunan_Fungsi}

\section{Praktikum}\label{praktikum-5}

\chapter{Konsep Integral Fungsi}\label{Konsep_Integral_Fungsi}

\section{Praktikum}\label{praktikum-6}

\chapter{Konsep Transenden Fungsi}\label{Konsep_Transenden_Fungsi}

\section{Praktikum}\label{praktikum-7}

\chapter{Penggunaan Transenden Fungsi}\label{Penggunaan_Transenden_Fungsi}

\section{Praktikum}\label{praktikum-8}

\chapter{Referensi}\label{referensi}

  \bibliography{book.bib,packages.bib}

\end{document}
