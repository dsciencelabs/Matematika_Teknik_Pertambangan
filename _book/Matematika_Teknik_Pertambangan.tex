% Options for packages loaded elsewhere
\PassOptionsToPackage{unicode}{hyperref}
\PassOptionsToPackage{hyphens}{url}
%
\documentclass[
]{book}
\usepackage{amsmath,amssymb}
\usepackage{iftex}
\ifPDFTeX
  \usepackage[T1]{fontenc}
  \usepackage[utf8]{inputenc}
  \usepackage{textcomp} % provide euro and other symbols
\else % if luatex or xetex
  \usepackage{unicode-math} % this also loads fontspec
  \defaultfontfeatures{Scale=MatchLowercase}
  \defaultfontfeatures[\rmfamily]{Ligatures=TeX,Scale=1}
\fi
\usepackage{lmodern}
\ifPDFTeX\else
  % xetex/luatex font selection
\fi
% Use upquote if available, for straight quotes in verbatim environments
\IfFileExists{upquote.sty}{\usepackage{upquote}}{}
\IfFileExists{microtype.sty}{% use microtype if available
  \usepackage[]{microtype}
  \UseMicrotypeSet[protrusion]{basicmath} % disable protrusion for tt fonts
}{}
\makeatletter
\@ifundefined{KOMAClassName}{% if non-KOMA class
  \IfFileExists{parskip.sty}{%
    \usepackage{parskip}
  }{% else
    \setlength{\parindent}{0pt}
    \setlength{\parskip}{6pt plus 2pt minus 1pt}}
}{% if KOMA class
  \KOMAoptions{parskip=half}}
\makeatother
\usepackage{xcolor}
\usepackage{longtable,booktabs,array}
\usepackage{calc} % for calculating minipage widths
% Correct order of tables after \paragraph or \subparagraph
\usepackage{etoolbox}
\makeatletter
\patchcmd\longtable{\par}{\if@noskipsec\mbox{}\fi\par}{}{}
\makeatother
% Allow footnotes in longtable head/foot
\IfFileExists{footnotehyper.sty}{\usepackage{footnotehyper}}{\usepackage{footnote}}
\makesavenoteenv{longtable}
\usepackage{graphicx}
\makeatletter
\newsavebox\pandoc@box
\newcommand*\pandocbounded[1]{% scales image to fit in text height/width
  \sbox\pandoc@box{#1}%
  \Gscale@div\@tempa{\textheight}{\dimexpr\ht\pandoc@box+\dp\pandoc@box\relax}%
  \Gscale@div\@tempb{\linewidth}{\wd\pandoc@box}%
  \ifdim\@tempb\p@<\@tempa\p@\let\@tempa\@tempb\fi% select the smaller of both
  \ifdim\@tempa\p@<\p@\scalebox{\@tempa}{\usebox\pandoc@box}%
  \else\usebox{\pandoc@box}%
  \fi%
}
% Set default figure placement to htbp
\def\fps@figure{htbp}
\makeatother
\setlength{\emergencystretch}{3em} % prevent overfull lines
\providecommand{\tightlist}{%
  \setlength{\itemsep}{0pt}\setlength{\parskip}{0pt}}
\setcounter{secnumdepth}{5}
\usepackage{booktabs}
\usepackage{lscape}

\usepackage[]{natbib}
\bibliographystyle{apalike}
\usepackage{bookmark}
\IfFileExists{xurl.sty}{\usepackage{xurl}}{} % add URL line breaks if available
\urlstyle{same}
\hypersetup{
  pdftitle={Matematika Teknik Pertambangan},
  pdfauthor={Bakti Siregar, M.Sc},
  hidelinks,
  pdfcreator={LaTeX via pandoc}}

\title{Matematika Teknik Pertambangan}
\author{Bakti Siregar, M.Sc}
\date{2024-07-10}

\begin{document}
\maketitle

{
\setcounter{tocdepth}{1}
\tableofcontents
}
\chapter*{Kata Pengantar}\label{kata-pengantar}
\addcontentsline{toc}{chapter}{Kata Pengantar}

Dengan mengucap syukur kepada Tuhan Yang Maha Esa, saya sangat senang dapat mempersembahkan eBook ``Matematika Tambang'' kepada para pembaca yang budiman. Buku ini merupakan hasil dari proses panjang, penelitian mendalam, serta dedikasi yang tiada henti. Harapan saya, eBook ini dapat memberikan manfaat yang signifikan bagi para pembaca, baik itu akademisi, praktisi, maupun masyarakat umum yang memiliki minat pada bidang ini.

Buku ini disusun dengan tujuan memberikan wawasan dan pengetahuan yang komprehensif mengenai konsep-konsep matematika yang relevan dan aplikatif dalam bidang pertambangan. Melalui buku ini, pembaca diharapkan dapat memahami konsep dasar kalkulus, aljabar linear, statistik, dan probabilitas yang diterapkan dalam pemodelan dan analisis data pertambangan.

\section*{Ringkasan Materi}\label{ringkasan-materi}
\addcontentsline{toc}{section}{Ringkasan Materi}

Adapun isi pembelajaran dalam modul ini adalah sebagai berikut:

\begin{itemize}
\tightlist
\item
  Kalkulus

  \begin{itemize}
  \tightlist
  \item
    Pengantar Matematika Tambang
  \item
    Bilangan Real
  \item
    Konsep Fungsi
  \item
    Konsep Limit Fungsi
  \item
    Konsep Turunan Fungsi
  \item
    Penggunaan Turunan
  \item
    Konsep Integral Fungsi
  \item
    Penggunaan Integral
  \item
    Konsep Fungsi Transenden
  \end{itemize}
\item
  Aljabar Linear

  \begin{itemize}
  \tightlist
  \item
    Matriks
  \item
    Determinan
  \item
    Sistem Persamaan Linear
  \item
    Eigenvalue
  \item
    Eigenvector
  \item
    Aplikasi dalam Pertambangan
  \end{itemize}
\item
  Optimisasi dalam Pertambangan

  \begin{itemize}
  \tightlist
  \item
    Pemrograman Linier
  \item
    Pemrograman Model Nonlinear
  \end{itemize}
\item
  Trigonometri dan Geometri

  \begin{itemize}
  \tightlist
  \item
    Trigonometri
  \item
    Geometri
  \end{itemize}
\end{itemize}

\section*{Penulis}\label{penulis}
\addcontentsline{toc}{section}{Penulis}

\begin{itemize}
\tightlist
\item
  \textbf{\href{https://www.linkedin.com/in/dsciencelabs/}{Bakti Siregar, M.Sc., CDS}} bekerja sebagai Dosen di Prodi Sains Data Institut Teknologi Sains Bandung. Beliau adalah meraih gelar Magister-nya dari Departemen Matematika Terapan (Applied Mathematics) National Sun Yat Sen University, Taiwan. Selain mengajar beliau juga pernah bekerja sebagai \textbf{Data Scientist Freelance} di perusahaan-perusahaan ternama seperti \href{https://www.jne.co.id/id/beranda}{JNE}, \href{https://www.samoragroup.co.id/home/en}{Samora Group}, \href{https://www.pertamina.com/}{Pertamina}, dan \href{https://ecgoevmoto.com/}{PT. Green City Traffic}. Beliau memiliki antusiasme khusus dalam mengerjakan proyek (mengajar) Big Data Analytics, Machine Learning, Optimisasi, dan Analisis Time Series di bidang keuangan dan investasi. Keahlian utama yang dimilikinya adalah bahasa pemrograman Statistik seperti R Studio dan Python. Beliau juga sudah terbiasa dalam mengaplikasikan sistem basis data MySQL/NoSQL untuk manajemen data, serta mahir dalam menggunakan tools Big Data seperti Spark dan Hadoop. Beberapa project beliau dapat dilihat di link berikut: \href{https://rpubs.com/dsciencelabs}{Rpubs}, \href{https://github.com/dsciencelabs}{Github}, \href{https://dsciencelabs.github.io/web/index.html}{Website}, dan \href{https://www.kaggle.com/baktisiregar/code}{Kaggle}.
\end{itemize}

\section*{Ucapan Terima Kasih}\label{ucapan-terima-kasih}
\addcontentsline{toc}{section}{Ucapan Terima Kasih}

Proses penulisan eBook ini tidak terlepas dari dukungan berbagai pihak. Saya ingin mengucapkan terima kasih kepada:

\begin{itemize}
\tightlist
\item
  \textbf{Keluarga} yang selalu memberikan dukungan moral dan semangat tanpa henti.
\item
  \textbf{Rekan-rekan dan Kolaborator} yang telah memberikan masukan, saran, dan kritik yang konstruktif.
\item
  Institusi dan Organisasi, Khususnya \textbf{\href{https://itsb.ac.id/}{ITSB}} yang telah menyediakan sumber daya dan fasilitas yang diperlukan selama proses penelitian dan penulisan.
\end{itemize}

Saya berharap eBook ini dapat menjadi referensi yang bermanfaat dan memberikan inspirasi serta pengetahuan baru bagi pembaca. Semoga Ebook ini dapat memenuhi ekspektasi dan kebutuhan para pembaca dan semoga ilmu yang disampaikan dapat bermanfaat bagi semua.

\section*{Masukan \& Saran}\label{masukan-saran}
\addcontentsline{toc}{section}{Masukan \& Saran}

Semua masukan dan tanggapan Anda sangat berarti bagi kami untuk memperbaiki Ebook Matematika Tambang ini kedepannya. Bagi para pembaca/pengguna yang ingin menyampaikan masukan dan tanggapan, dipersilahkan melalui kontak dibawah ini!

\textbf{Email:} \href{mailto:dsciencelabs@outlook.com}{\nolinkurl{dsciencelabs@outlook.com}}

\chapter{Pengantar}\label{pengantar}

Matematika Teknik Pertambangan adalah bidang studi yang memadukan konsep-konsep terapan ilmu matematika dalam berbagai ruang lingkup teknik pertambangan. Pada konteks ini, matematika digunakan untuk menyelesaikan berbagai masalah yang terkait dengan eksplorasi, ekstraksi, dan pengolahan mineral.

\section{Konsep Matematika}\label{konsep-matematika}

Matematika memainkan peran penting dalam berbagai aspek teknik pertambangan. Berikut adalah beberapa konsep matematika yang sangat penting dalam bidang ini:

\subsection{Kalkulus (Diferensial dan Integral)}\label{kalkulus-diferensial-dan-integral}

Kalkulus digunakan dalam perhitungan volume, optimasi, dan analisis kestabilan.

\begin{itemize}
\tightlist
\item
  \textbf{Perhitungan Volume}: Menggunakan integral untuk menghitung volume cadangan mineral.
\item
  \textbf{Optimasi Rute Penambangan}: Menggunakan derivatif untuk menemukan titik maksimum atau minimum fungsi biaya atau keuntungan.
\item
  \textbf{Analisis Kestabilan Lereng}: Persamaan diferensial digunakan untuk memodelkan kestabilan lereng tambang.
\end{itemize}

Adapun teori Diferensial dan Integral yang digunakan adalah sebagai berikut:

\begin{itemize}
\tightlist
\item
  \textbf{Integral Ganda}: Digunakan dalam perhitungan volume.
\item
  \textbf{Persamaan Diferensial}: Digunakan untuk model transportasi panas dan aliran fluida.
\end{itemize}

\subsection{Aljabar Linier}\label{aljabar-linier}

Aljabar linier sangat penting dalam pemodelan data geologi dan analisis seismik.

\begin{itemize}
\tightlist
\item
  \textbf{Analisis Data Seismik}: Menggunakan matriks untuk memproses data gelombang seismik.
\item
  \textbf{Pengolahan Citra Geologi}: Transformasi linear untuk memanipulasi citra geologi.
\item
  \textbf{Model Geostatistik}: Matriks kovarians digunakan dalam pemodelan distribusi mineral.
\end{itemize}

Adapun Aljabar Linier yang digunakan adalah sebagai berikut:

\begin{itemize}
\tightlist
\item
  \textbf{Matriks dan Vektor}: Digunakan dalam pemodelan data multidimensi.
\item
  \textbf{Dekomposisi Nilai Singular (SVD)}: Digunakan untuk mengurangi dimensi data.
\end{itemize}

\subsection{Teori Optimasi}\label{teori-optimasi}

Optimasi digunakan untuk perencanaan dan operasi tambang yang efisien.

\begin{itemize}
\tightlist
\item
  \textbf{Rencana Penambangan Optimal}: Menentukan jadwal produksi yang memaksimalkan keuntungan.
\item
  \textbf{Alokasi Sumber Daya}: Optimasi penggunaan mesin dan tenaga kerja.
\item
  \textbf{Jadwal Produksi}: Menggunakan algoritma optimasi untuk merencanakan produksi yang efisien.
\end{itemize}

Adapun metode Optimasi yang digunakan adalah sebagai berikut:

\begin{itemize}
\tightlist
\item
  \textbf{Pemrograman Linier}: Untuk masalah optimasi dengan kendala linier.
\item
  \textbf{Pemrograman Non-Linier}: Untuk masalah dengan fungsi objektif non-linier.
\item
  \textbf{Algoritma Genetika}: Metode heuristik untuk menemukan solusi mendekati optimal.
\end{itemize}

\subsection{Geometri dan Trigonometri}\label{geometri-dan-trigonometri}

Geometri dan trigonometri digunakan dalam survei tambang dan perencanaan desain.

\begin{itemize}
\tightlist
\item
  \textbf{Survei Tambang}: Pengukuran dan pemetaan wilayah tambang.
\item
  \textbf{Perencanaan Desain Tambang}: Merancang struktur tambang yang aman dan efisien.
\item
  \textbf{Analisis Struktur Geologi}: Menggunakan geometri untuk memahami bentuk dan orientasi struktur geologi.
\end{itemize}

Adapun ilmu Geometri dan Trigonometri yang digunakan adalah sebagai berikut:

\begin{itemize}
\tightlist
\item
  \textbf{Pengukuran Sudut dan Panjang}: Trigonometri digunakan untuk menentukan jarak dan sudut dalam survei.
\item
  \textbf{Transformasi Koordinat}: Mengubah data dari satu sistem koordinat ke sistem lain.
\end{itemize}

\section{Terapan Matematika}\label{terapan-matematika}

\subsection{Kalkulus (Diferensial dan Integral)}\label{kalkulus-diferensial-dan-integral-1}

\subsubsection*{Integral}\label{integral}
\addcontentsline{toc}{subsubsection}{Integral}

Integral tak tentu:

\[\int f(x) \, dx = F(x) + C\]

di mana \(F(x)\) adalah antiturunan dari \(f(x)\) dan \(C\) adalah konstanta integrasi.

Integral tertentu:

\[
  \int_a^b f(x) \, dx = F(b) - F(a)
\]

di mana \(F(x)\) adalah antiturunan dari \(f(x)\).

Menghitung volume mineral dalam sebuah tambang dengan bentuk paraboloid:

\[
  V = \int_0^h \pi r^2 \, dz = \pi \int_0^h \left( \frac{R}{h} z \right)^2 \, dz
\]

\[
  V = \pi \frac{R^2}{h^2} \int_0^h z^2 \, dz = \pi \frac{R^2}{h^2} \left[ \frac{z^3}{3} \right]_0^h = \pi \frac{R^2}{h^2} \cdot \frac{h^3}{3} = \frac{1}{3} \pi R^2 h
\]
Jadi, volume mineral adalah \(\frac{1}{3} \pi R^2 h\).

\subsubsection*{Persamaan Diferensial}\label{persamaan-diferensial}
\addcontentsline{toc}{subsubsection}{Persamaan Diferensial}

Persamaan diferensial linear orde pertama:

\[
  \frac{dy}{dx} + P(x)y = Q(x)
\]

Menentukan laju perubahan konsentrasi gas dalam sebuah tambang. Misalkan persamaan diferensialnya adalah:

\[
  \frac{dC}{dt} + 0.1C = 2
\]
Ini adalah persamaan diferensial linier dengan \(P(t) = 0.1\) dan \(Q(t) = 2\). Solusinya adalah:

\[
  C(t) = e^{-0.1t} \left( \int 2 e^{0.1t} \, dt \right) = e^{-0.1t} \left( \frac{2}{0.1} e^{0.1t} + C_1 \right) = 20 + C_1 e^{-0.1t}
\]

di mana \(C_1\) adalah konstanta integrasi.

\subsubsection*{Derivatif}\label{derivatif}
\addcontentsline{toc}{subsubsection}{Derivatif}

Turunan fungsi:

\[
  \frac{dy}{dx} = \lim_{\Delta x \to 0} \frac{f(x + \Delta x) - f(x)}{\Delta x}
\]

Menentukan laju perubahan kedalaman tambang dengan waktu. Misalkan kedalaman tambang \(d(t)\) diberikan oleh:

\[
  d(t) = 5t^2 + 3t + 10
\]
Maka, laju perubahan kedalaman adalah:

\[
  \frac{dd}{dt} = \frac{d}{dt} (5t^2 + 3t + 10) = 10t + 3
\]
Jadi, laju perubahan kedalaman pada waktu \(t = 2\) adalah:

\[
  \frac{dd}{dt} \bigg|_{t=2} = 10(2) + 3 = 23
\]
Jadi, laju perubahan kedalaman pada waktu 2 adalah 23 meter per satuan waktu.

\subsection{Aljabar Linier}\label{aljabar-linier-1}

\subsubsection*{Matriks}\label{matriks}
\addcontentsline{toc}{subsubsection}{Matriks}

Dalam teknik pertambangan, matriks memiliki berbagai aplikasi yang penting dalam analisis data geologis dan perencanaan tambang. Misalkan kita memiliki data sederhana tentang kadar mineral (M) dan densitas batuan (D) di suatu area tambang yang direpresentasikan dalam bentuk matriks sebagai berikut:

\[
X = \begin{bmatrix}
M_1 & D_1 \\
M_2 & D_2 \\
M_3 & D_3 \\
\end{bmatrix}
\]

Kita ingin menghitung matriks variansi-kovariansi untuk menentukan keterkaitan antara kadar mineral dan densitas batuan. Matriks variansi-kovariansi \(\Sigma\) dapat dihitung dengan rumus:

\[ 
\Sigma = \frac{1}{n-1} \sum_{i=1}^{n} (\mathbf{x}_i - \bar{\mathbf{x}}) (\mathbf{x}_i - \bar{\mathbf{x}})^T 
\]

di mana \(\mathbf{x}_i\) adalah vektor observasi ke-i, \(\bar{\mathbf{x}}\) adalah vektor rata-rata dari data, dan \(n\) adalah jumlah sampel.

Langkah-langkah Perhitungan:

\begin{itemize}
\item
  \textbf{Hitung Rata-Rata}: \(\bar{\mathbf{x}} = \frac{1}{n} \sum_{i=1}^{n} \mathbf{x}_i\)
\item
  \textbf{Hitung Variansi-Kovariansi}:
\end{itemize}

\[
\Sigma = \frac{1}{2} \begin{bmatrix}
\text{Var}(M) & \text{Cov}(M,D) \\
\text{Cov}(M,D) & \text{Var}(D) \\
\end{bmatrix}
\]

Misalnya, jika data yang diamati adalah:

\[
X = \begin{bmatrix}
3 & 5 \\
2 & 4 \\
4 & 6 \\
\end{bmatrix}
\]

Maka:

\begin{itemize}
\item
  Rata-rata \(\bar{\mathbf{x}} = \begin{bmatrix} 3 & 5 \end{bmatrix}\)
\item
  Variansi-Kovariansi \(\Sigma = \frac{1}{2} \begin{bmatrix} 1 & 1 \\ 1 & 1 \end{bmatrix}\)
\end{itemize}

Ini menunjukkan bahwa ada keterkaitan positif antara kadar mineral dan densitas batuan dalam data sederhana ini.

Dalam contoh sederhana ini, kita melihat bagaimana matriks variansi-kovariansi digunakan untuk menganalisis hubungan antara variabel geologis dalam teknik pertambangan. Penggunaan matriks seperti ini membantu dalam pemahaman lebih dalam tentang struktur data geologis yang dapat digunakan untuk pengambilan keputusan lebih lanjut dalam manajemen tambang.

\subsubsection*{Vektor}\label{vektor}
\addcontentsline{toc}{subsubsection}{Vektor}

Dalam teknik pertambangan, vektor digunakan untuk merepresentasikan dan menganalisis berbagai informasi penting seperti koordinat geografis dan orientasi struktur geologi. Berikut adalah contoh penghitungan manual norma vektor (magnitude) dan dot product (produk titik) dari dua vektor:

Misalkan kita memiliki dua vektor dalam ruang tiga dimensi:

\[ \mathbf{v}_1 = (3, -2, 1) \]
\[ \mathbf{v}_2 = (1, 1, 2) \]

Hitung Norma vektor \(\mathbf{v} = (a, b, c)\) didefinisikan sebagai:

\[ \| \mathbf{v} \| = \sqrt{a^2 + b^2 + c^2} \]

\textbf{Untuk \(\mathbf{v}_1\):}

\[ \| \mathbf{v}_1 \| = \sqrt{3^2 + (-2)^2 + 1^2} \]
\[ \| \mathbf{v}_1 \| = \sqrt{9 + 4 + 1} \]
\[ \| \mathbf{v}_1 \| = \sqrt{14} \approx 3.74 \]

\textbf{Untuk \(\mathbf{v}_2\):}

\[ \| \mathbf{v}_2 \| = \sqrt{1^2 + 1^2 + 2^2} \]
\[ \| \mathbf{v}_2 \| = \sqrt{1 + 1 + 4} \]
\[ \| \mathbf{v}_2 \| = \sqrt{6} \approx 2.45 \]

Hitung Dot product dari dua vektor \(\mathbf{v}_1 = (a, b, c)\) dan \(\mathbf{v}_2 = (d, e, f)\) didefinisikan sebagai:

\[ \mathbf{v}_1 \cdot \mathbf{v}_2 = ad + be + cf \]

\textbf{Untuk \(\mathbf{v}_1\) dan \(\mathbf{v}_2\):}

\[ \mathbf{v}_1 \cdot \mathbf{v}_2 = 3 \cdot 1 + (-2) \cdot 1 + 1 \cdot 2 \]
\[ \mathbf{v}_1 \cdot \mathbf{v}_2 = 3 - 2 + 2 \]
\[ \mathbf{v}_1 \cdot \mathbf{v}_2 = 3 \]

\subsection{Teori Optimasi}\label{teori-optimasi-1}

\subsubsection*{Pemrograman Linier}\label{pemrograman-linier}
\addcontentsline{toc}{subsubsection}{Pemrograman Linier}

Fungsi objektif:

\[
  \text{Maksimalkan} \quad Z = c_1 x_1 + c_2 x_2 + \ldots + c_n x_n
\]
di bawah kendala:

\[
  \begin{cases}
  a_{11}x_1 + a_{12}x_2 + \ldots + a_{1n}x_n \le b_1 \\
  a_{21}x_1 + a_{22}x_2 + \ldots + a_{2n}x_n \le b_2 \\
  \vdots \\
  a_{m1}x_1 + a_{m2}x_2 + \ldots + a_{mn}x_n \le b_m \\
  x_1, x_2, \ldots, x_n \ge 0
  \end{cases}
\]

Misalkan kita ingin memaksimalkan produksi dua jenis mineral \(M_1\) dan \(M_2\) dengan kendala biaya dan waktu. Fungsi objektif dan kendalanya adalah:

\[
\text{Maksimalkan} \quad Z = 40x_1 + 30x_2
\]
\[
\begin{cases}
2x_1 + 3x_2 \le 60 \quad \text{(biaya)} \\
4x_1 + 2x_2 \le 80 \quad \text{(waktu)} \\
x_1, x_2 \ge 0
\end{cases}
\]

Dengan menggunakan metode Simplex, kita bisa menemukan solusi optimal \(x_1 = 10\) dan \(x_2 = 0\), dengan nilai objektif \(Z = 400\).

\subsubsection*{Pemrograman Non-Linier}\label{pemrograman-non-linier}
\addcontentsline{toc}{subsubsection}{Pemrograman Non-Linier}

Fungsi objektif:

\[
\text{Maksimalkan} \quad f(x_1, x_2, \ldots, x_n)
\]
di bawah kendala:

\[
\begin{cases}
g_i(x_1, x_2, \ldots, x_n) \le 0, \quad i = 1, 2, \ldots, m \\
h_j(x_1, x_2, \ldots, x_n) = 0, \quad j = 1, 2, \ldots, p
\end{cases}
\]

Optimasi produksi tambang dengan fungsi objektif:

\[
f(x_1, x_2) = x_1^2 + x_2^2 + x_1x_2
\]
di bawah kendala:

\[
\begin{cases}
x_1^2 + x_2^2 \le 10 \\
x_1 + x_2 \le 5 \\
x_1, x_2 \ge 0
\end{cases}
\]
Menggunakan metode optimasi non-linier seperti Lagrange atau metode numerik, kita bisa menemukan solusi optimal.

\subsection{Trigonometri dan Geometri}\label{trigonometri-dan-geometri}

\subsubsection*{Trigonometri}\label{trigonometri}
\addcontentsline{toc}{subsubsection}{Trigonometri}

Identitas dasar:

\[
\sin^2 \theta + \cos^2 \theta = 1
\]

Dalam menentukan sudut elevasi sebuah lereng tambang:

\[
  \sin \theta = \frac{\text{tinggi}}{\text{panjang miring}} = \frac{3}{5} = 0.6
\]

\[
\cos \theta = \sqrt{1 - \sin^2 \theta} = \sqrt{1 - 0.6^2} = \sqrt{1 - 0.36} = \sqrt{0.64} = 0.8
\]

\subsubsection*{Pengukuran Sudut}\label{pengukuran-sudut}
\addcontentsline{toc}{subsubsection}{Pengukuran Sudut}

Panjang busur \(s\):

\[
s = r \theta
\]

di mana \(r\) adalah jari-jari lingkaran dan \(\theta\) adalah sudut dalam radian.

Mengukur panjang busur dari sebuah tambang berbentuk lingkaran dengan jari-jari 50 meter dan sudut 30 derajat (\(\frac{\pi}{6}\) radian):

\[
s = 50 \times \frac{\pi}{6} \approx 50 \times 0.5236 = 26.18 \text{ meter}
\]

\subsubsection*{Transformasi Koordinat}\label{transformasi-koordinat}
\addcontentsline{toc}{subsubsection}{Transformasi Koordinat}

Transformasi dari koordinat kartesian ke koordinat polar:

\[
\begin{cases}
r = \sqrt{x^2 + y^2} \\
\theta = \tan^{-1} \left( \frac{y}{x} \right)
\end{cases}
\]

Menentukan koordinat polar dari titik tambang dengan koordinat kartesian \((3, 4)\):

\[
r = \sqrt{3^2 + 4^2} = \sqrt{9 + 16} = \sqrt{25} = 5
\]
\[
\theta = \tan^{-1} \left( \frac{4}{3} \right) \approx 0.927 \text{ radian}
\]

\chapter{Bilangan Real}\label{Bilangan_Real}

Bilangan real adalah himpunan semua bilangan rasional dan irasional. Himpunan ini meliputi bilangan bulat, bilangan pecahan, dan bilangan desimal yang dapat dinyatakan secara tak terbatas. Mari kita bahas definisi, sifat, jenis, rumus, dan contoh materi bilangan real secara lebih mendetail.

\section{Definisi Bilangan Real}\label{definisi-bilangan-real}

Bilangan real adalah himpunan semua bilangan rasional (termasuk bilangan bulat dan pecahan) serta bilangan irasional. Simbol untuk himpunan bilangan real adalah \(\mathbb{R}\).

\section{Sifat-sifat Bilangan Real}\label{sifat-sifat-bilangan-real}

\begin{enumerate}
\def\labelenumi{\arabic{enumi}.}
\tightlist
\item
  \textbf{Komutatif dan Asosiatif:} Operasi penjumlahan dan perkalian pada bilangan real bersifat komutatif dan asosiatif.
\item
  \textbf{Distributif:} Operasi perkalian distributif terhadap penjumlahan.
\item
  \textbf{Aditif Invers:} Setiap bilangan real memiliki aditif invers, yaitu bilangan yang jika ditambahkan akan menghasilkan nol.
\item
  \textbf{Multiplikatif Invers:} Kecuali untuk nol, setiap bilangan real memiliki multiplikatif invers.
\end{enumerate}

\section{Jenis-jenis Bilangan Real}\label{jenis-jenis-bilangan-real}

\begin{enumerate}
\def\labelenumi{\arabic{enumi}.}
\item
  \textbf{Bilangan Bulat ( \(\mathbb{Z}\) ):} Bilangan bulat adalah bilangan positif dan negatif yang tidak memiliki pecahan atau bagian desimal.
\item
  \textbf{Bilangan Rasional ( \(\mathbb{Q}\) ):} Bilangan rasional adalah bilangan yang dapat dinyatakan sebagai pecahan \(\frac{a}{b}\) di mana \(a\) dan \(b\) adalah bilangan bulat dan \(b \neq 0\).
\item
  \textbf{Bilangan Irrasional:} Bilangan irasional tidak dapat dinyatakan dalam bentuk pecahan dan memiliki ekspansi desimal yang tidak berakhir atau berulang.
\end{enumerate}

\section{Aplikasi dalam Matematika Teknik Pertambangan}\label{aplikasi-dalam-matematika-teknik-pertambangan}

\begin{enumerate}
\def\labelenumi{\arabic{enumi}.}
\tightlist
\item
  \textbf{Perhitungan Kadar Mineral:}

  \begin{itemize}
  \tightlist
  \item
    Estimasi dan analisis kadar mineral di suatu area tambang menggunakan bilangan real untuk menghitung rata-rata dan variabilitasnya.
  \end{itemize}
\item
  \textbf{Pemodelan Geostatistik:}

  \begin{itemize}
  \tightlist
  \item
    Penggunaan bilangan real dalam pemodelan geostatistik untuk memprediksi distribusi dan kuantitas mineral di dalam tanah.
  \end{itemize}
\item
  \textbf{Analisis Biaya dan Produksi:}

  \begin{itemize}
  \tightlist
  \item
    Perhitungan biaya operasional per unit produksi menggunakan bilangan real untuk evaluasi ekonomi dari suatu tambang.
  \end{itemize}
\item
  \textbf{Pemodelan Hidrogeologi:}

  \begin{itemize}
  \tightlist
  \item
    Penggunaan bilangan real untuk menganalisis aliran air bawah tanah dan dampak lingkungan dalam pengelolaan air di tambang.
  \end{itemize}
\end{enumerate}

\section{Contoh Kasus}\label{contoh-kasus}

Sebagai contoh konkret, dalam analisis eksplorasi tambang, seorang geolog dapat menggunakan data kedalaman pengeboran (bilangan real) untuk mengevaluasi potensi suatu deposit mineral. Dengan memahami distribusi dan karakteristik bilangan real, mereka dapat membuat perkiraan yang lebih akurat tentang potensi hasil tambang dan kebutuhan operasional.

\section{Kesimpulan}\label{kesimpulan}

Pemahaman yang baik tentang bilangan real penting dalam matematika teknik pertambangan karena mendukung berbagai analisis dan perhitungan yang esensial dalam pengelolaan dan pengembangan tambang. Dengan menggunakan konsep bilangan real ini, para profesional dapat meningkatkan efisiensi operasional dan mengoptimalkan keputusan teknis dalam industri ini.

\chapter{Konsep Fungsi}\label{Konsep_Fungsi}

\section{Praktikum}\label{praktikum}

\chapter{Konsep Limit Fungsi}\label{Konsep_Limit_Fungsi}

\section{Praktikum}\label{praktikum-1}

\chapter{Konsep Turunan}\label{Konsep_Turunan}

\section{Praktikum}\label{praktikum-2}

\chapter{Penggunaan Turunan Fungsi}\label{Penggunaan_Turunan_Fungsi}

\section{Praktikum}\label{praktikum-3}

\chapter{Konsep Integral Fungsi}\label{Konsep_Integral_Fungsi}

\section{Praktikum}\label{praktikum-4}

\chapter{Konsep Transenden Fungsi}\label{Konsep_Transenden_Fungsi}

\section{Praktikum}\label{praktikum-5}

\chapter{Penggunaan Transenden Fungsi}\label{Penggunaan_Transenden_Fungsi}

\section{Praktikum}\label{praktikum-6}

\chapter{Referensi}\label{referensi}

  \bibliography{book.bib,packages.bib}

\end{document}
